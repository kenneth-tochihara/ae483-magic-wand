\documentclass[conf]{new-aiaa}
%\documentclass[journal]{new-aiaa} for journal papers
\usepackage[utf8]{inputenc}

\usepackage{graphicx}
\usepackage{amsmath}
\usepackage[version=4]{mhchem}
\usepackage{siunitx}
\usepackage{longtable,tabularx}
\usepackage{float}
\setlength\LTleft{0pt} 

\title{Wii-mote Controlled Graphical User Interface for Crazyflie Drone}

\author{Erika N. Jarosch\footnote{Student}, George V. Petrov\footnote{Student}, Justin L. Roskamp\footnote{Student}, and Kenneth T. Tochihara\footnote{Student}}
\affil{University of Illinois at Urbana-Champaign, Urbana, IL, 61820}

\begin{document}

\maketitle

\begin{abstract} % Erika
The goal of this project is to design and implement a multimodal control interface between a Nintendo Wii remote and a Crazyflie 2.0 drone. The Wii Remote serves as an input device to a macOS device via DarwiinRemote. This input is then used with Python package Tkinter to convert Wii remote inputs to positional inputs, which can then be passed through one of three anticipated modes: Copycat, Spellcaster, and Live. Each mode interprets Wii remote inputs differently and then passes the mode-determined flight plan to the Crazyflie via the input structures developed in the AE 483 course at the University of Illinois at Urbana-Champaign. This goal is motivated by the desire to inspire others with innovative and interactive aerobatics.

\end{abstract}

\section{Nomenclature}
% dynamic- all :)

\section{Introduction}
% Kenneth (v simple, mostly copy)

\section{Methods}
% what we used: equations, software and packages

    \subsection{Python Packages}
        
        % tkinter, numpy, time,  (Erika)
        The basis for overall development of the GUI draws heavily upon the tkinter package. Tkinter, an abbreviation for the Tk interface, is the Python interface to the Tcl/Tk GUI toolkit. This toolkit is an amalgam of modules with specific functionalities. Tcl is a dynamic interpreted programming language with a library that utilizes a C interface to command instances of a Tcl interpreter. In tandem with Tkinter, multitasking capabilities are functional via threading. One of the Tcl packages, Tk, allowed the group to customize buttons for the interface. 
        
        % crazyflie client (George)
        
        % threading (Kenneth)
        Threading ma
        
        % gui.py: pixel conversions and normalization set-up (Erika)
        
        % client.py:  (Justin)
    
    % client firmware and Wii-mote (George)
    \subsection{Client firmware}
    \subsection{Wii-mote}


\section{Design}
% how we implemented code

% how we used hardware (Justin)
    \subsection{Infrared Sensor Bar}
    
        The original intent of the project was to utilize the Wii Sensor Bar in tandem with the Wii Remote as a positional input system, similar to a mouse. With the Wii, the Sensor Bar allows the user to calibrate a pointing system that can be used to point the remote at specific locations onscreen.
        
        The "Sensor Bar" is a technical misnomer. The bar itself does not do any sensing and is instead an array of passive infrared (IR) LEDs that the Wii Remote senses via an infrared camera on the end of the device. This provides a reference in the field of view of the remote that can be used to determine where the remote is pointing.
        
        In fact, our previous experience with the remote has demonstrated that the Sensor Bar is not required for pointing operations---any appropriately bright infrared source can work, including flames (such as from candles). With this knowledge and the fact that the Wii Sensor Bar has a proprietary connector that is powered via the Wii console, we considered developing or purchasing a USB-powered infrared LED array to simplify the use of our system. Ultimately, this was not required.
        
        Initial testing of the Wii Remote with infrared sources did not appear to provide any positional data to the DarwiinRemote program on macOS. While troubleshooting, we realized that a simpler input method with Wii MotionPlus was possible, and the infrared implementation was effectively abandoned. We believe the infrared approach is possible and worth pursuing in the future, though it was unnecessarily complex in the light of MotionPlus-enabled control. 
    
    \subsection{Wii MotionPlus} %GEORGE PLEASE ADD HOW THE FUCK GLOVEPIE OPERATES (na)
    
        Wii MotionPlus is an expansion module for the Wii Remote that adds a gyroscope, allowing the controller to detect rotations in addition to linear acceleration and pointing from the built-in accelerometer and IR sensor. Later versions of the Wii Remote, namely the Wii Remote Plus, incorporated the MotionPlus technology inside one device.
        
        As compatibility issues with DarwiinRemote arose with the mid-project upgrade to macOS Monterey, troubleshooting and searches for alternative software led to the use of GlovePIE on Windows 10.
    
    % Wii-motion plus
    

    % walk through code set up and implementation
    \subsection{Code Implementation}
    
        \subsubsection{Python}
        % gui.py (Justin)
        % client.py (Erika)
        % flight data folder (Kenneth)
        % main.py and run file (Kenneth, ???)
        
        \subsubsection{CrazyFlie Firmware}
        % custom controller, og observer (George)

\section{Results and Discussion}
% WHAT and WHY functionalities worked/didn't work, iteration (Erika)

% ABORTIONS (Kenneth)

% Not on mac (Nintendo sux for proprietary software) (George)

% observer traced (Justin)

\section{Conclusion}
% George

\section{Acknowledgements}
yo mama, and cookies from bretl thenk u

\appendix

\section{Group Member Contributions} \label{apx:contributions}
    \begin{table}[H]
        \begin{center}
        \setstretch{1}
        \caption{\textbf{Contributions Table}}
        \begin{tabular}{ | p{2in} | p{4in}| } 
            \hline
            \textbf{Group Member} & \textbf{Contribution to Project} \\  \hline
            Erika Jarosch & Flight path code development and interfacing, co-authored Methods, Design, Results, Developed Jupyter Notebook\\ \hline
            George Petrov & Wii Remote implentation, documentation, authored Wii remote section on report\\ \hline
            Justin Roskamp & Interpreted real-world performance and configurations, co-authored Introduction, Design, Results, Conclusion   \\ \hline
            Kenneth Tochihara & GUI and client interface, repository management, documentation, co-authored code-implementation sections on report\\ \hline
        \end{tabular}
        \end{center}
    \end{table}

\section{Code Repository}
    GitHub Repository: \url{https://github.com/ktt3/ae483-magic-wand}



\end{document}