\documentclass[conf]{new-aiaa}
%\documentclass[journal]{new-aiaa} for journal papers
\usepackage[utf8]{inputenc}

\usepackage{graphicx}
\usepackage{amsmath}
\usepackage[version=4]{mhchem}
\usepackage{siunitx}
\usepackage{longtable,tabularx}
\setlength\LTleft{0pt} 

\title{Multimodal Control of Crazyflie Drone with Wii Remote}

\author{First A. Author\footnote{Insert Job Title, Department Name, Address/Mail Stop, and AIAA Member Grade (if any) for first author.} and Second B. Author Jr.\footnote{Insert Job Title, Department Name, Address/Mail Stop, and AIAA Member Grade (if any) for second author.}}
\affil{Business or Academic Affiliation 1, City, State, Zip Code}

\begin{document}

\maketitle

\begin{abstract}
The goal of this project is to design and implement a multimodal control interface between a Wii remote and a Crazyflie drone. which will simultaneously move at least two CrazyFlie quadcopters throughout their flight. The controller will be a handheld camera device, such as a Wii remote. The group will utilize two light sources to allow the remote sensing capabilities. Ultimately, the sensing of this remote will contribute to our ability to control the flight patterns and location of our drones. This goal is motivated by the desire to inspire via innovative aerobatics, similar to the drone show that was displayed at an event during the 75th anniversary of the Aerospace Engineering department at the University of Illinois.  Along a similar vein, just as Harry Potter wands in the Wizarding World theme park are used to control objects, a similar philosophy could be used to “cast spells” on drones—the sky is the limit.  Developing intuitive controls beyond those provided by a simple directional pad or buttons can unlock new ways of interacting with these aircraft. Our project will focus primarily on the first order implementation of these controls as a proof of concept. Time permitting, the group would find it interesting to show the drone complete a user specified task, such as solving a 3D maze or obstacle course. A sufficient controller will ensure that the drone remains within stable enough margins while completing these tasks. Our milestones will be to first implement a similar “move_smooth”  function with coordinate inputs on a 2D plane (Nov 1). Then  we will implement an infrared position system (IPS) to return a set of coordinate ranges set by the remote. This will include providing a GUI of the path determined (Nov 8). Finally we will  integrate IPS with the move_smooth variant and test the controller (Nov 15).
\end{abstract}

\section{Nomenclature}

    % {\renewcommand\arraystretch{1.0}
    % \noindent\begin{longtable*}{@{}l @{\quad=\quad} l@{}}
    % $A$  & amplitude of oscillation \\
    % $a$ &    cylinder diameter \\
    % $C_p$& pressure coefficient \\
    % $Cx$ & force coefficient in the \textit{x} direction \\
    % $Cy$ & force coefficient in the \textit{y} direction \\
    % c   & chord \\
    % d$t$ & time step \\
    % $Fx$ & $X$ component of the resultant pressure force acting on the vehicle \\
    % $Fy$ & $Y$ component of the resultant pressure force acting on the vehicle \\
    % $f, g$   & generic functions \\
    % $h$  & height \\
    % $i$  & time index during navigation \\
    % $j$  & waypoint index \\
    % $K$  & trailing-edge (TE) nondimensional angular deflection rate
    % \end{longtable*}}

\section{Introduction}

\section{Milestones}

    \subsection{Week of November 1}
    
    \subsection{Week of November 8}
    
    \subsection{Week of November 15}
    
    \subsection{Week of November 29}

\section{Conclusion}

    A conclusion section is not required, though it is preferred. Although a conclusion may review the main points of the paper, do not replicate the abstract as the conclusion. A conclusion might elaborate on the importance of the work or suggest applications and extensions. \textit{Note that the conclusion section is the last section of the paper that should be numbered. The appendix (if present), acknowledgment, and references should be listed without numbers.}


\section*{Appendix}

    An Appendix, if needed, should appear before the acknowledgments.

\section*{Acknowledgments}

    The authors thank their teaching assistant, Travis Zook, and the lab m

\bibliography{sample}

\end{document}